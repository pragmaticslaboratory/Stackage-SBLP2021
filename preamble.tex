
\usepackage[T1]{fontenc}
\usepackage[utf8]{inputenc}

%---- PACKAGES
\usepackage{amssymb}
\usepackage{mathrsfs}
\usepackage{hyperref}
\usepackage[plain]{fancyref}
\usepackage{ifdraft}
\usepackage{subcaption}
\let\labelindent\relax
\usepackage[inline]{enumitem}
\usepackage{xcolor}
\usepackage{booktabs}
\usepackage{xspace}
\usepackage[final]{listings}
\usepackage{acronym}
\usepackage{url}
\urlstyle{sf}
\usepackage{amsmath}
\usepackage{csquotes}
\usepackage{natbib}
\usepackage{parcolumns}
\usepackage{bold-extra}
%\usepackage{cleveref}


\usepackage{etoolbox}
\makeatletter
\patchcmd{\@makecaption}
  {\scshape}
  {}
  {}
  {}
\patchcmd{\@makecaption}
  {\\}
  {.\ }
  {}
  {}


%\let\refsection\relax


%\usepackage[%backend=bibtex,
%%   style=trad-abbrv,
%%   maxnames=5,
%%   firstinits=true,
%%   hyperref=true,
%   natbib=true,
%   url=false,
%   doi=false,
%   defernumbers]{biblatex}
%%
%
%----[ Biber ]----
%\addbibresource[datatype=bibtex]{local.bib}
%\addbibresource[datatype=bibtex]{bib/general.bib}
%\addbibresource[datatype=bibtex]{bib/compsci.bib}
%\addbibresource[datatype=bibtex]{bib/learning.bib}
%\nocite{*}
%
%\newbibmacro{name:newformat}{%
%    \printnames{authors}
%%   \textbf{\namepartfamily}  % #1->\namepartfamily, #2->\namepartfamilyi
%%   \textbf{\namepartgiven}   % #3->\namepartgiven,  #4->\namepartgiveni
%%   [prefix: \namepartprefix] % #5->\namepartprefix, #6->\namepartprefixi
%%   [suffix: \namepartsuffix] % #7->\namepartsuffix, #8->\namepartsuffixi
%}
%
%\DeclareNameFormat{newformat}{%
%  \usebibmacro{name:newformat}{\textbf{#1}}{\textbf{#3}}{#5}{#7}%
%  \usebibmacro{name:andothers}%
%%  \nameparts{#1}% split the name data, will not be necessary in future versions
%%  \usebibmacro{name:newformat}%
%%  \usebibmacro{name:andothers}%
%}
%
%\AtEveryBibitem
%{
%   \clearlist{address}
%   \clearfield{date}
%   \clearfield{doi}
%   \clearfield{eprint}
%   \clearfield{isbn}
%   \clearfield{issn}
%   \clearfield{month}
%   \clearfield{note}
%%   \clearfield{pages}
%   \clearlist{location}
%%   \clearfield{series}
%   \clearfield{url}
%   \clearname{editor}
%   \ifentrytype{inproceedings}
%     {\clearfield{day}
%      \clearfield{month}
%      \clearfield{volume}}{}
%}
%
%\DeclareFieldFormat*{title}{\textsl{#1}\isdot}
%\DeclareFieldFormat*{journaltitle}{#1}
%\DeclareFieldFormat*{booktitle}{#1}
%
%\renewbibmacro{in:}{} % supress 'In: ' form
%
%\DeclareSourcemap
% {\maps[datatype=bibtex,overwrite]
%   {% Tag entries (through keywords)
%    \map
%      {\step[fieldsource=booktitle,
%       match=\regexp{[Pp]roceedings}, replace={Proc.}]}
%        \map
%      {\step[fieldsource=booktitle,
%       match=\regexp{[Ii]nternational\s+[Cc]onference}, replace={Intl. Conf.}]}
%    \map
%      {\step[fieldsource=journal,
%       match=\regexp{[Jj]ournal}, replace={Jour.}]}
%    \map
%      {\step[fieldsource=journal,
%       match=\regexp{[Tt]ransactions}, replace={Trans.}]}
%    \map
%      {\step[fieldsource=booktitle,
%       match=\regexp{[Pp]roceedings\s+of\s+the.+[Ee]uropean\s+[Cc]onference\s+in}, replace={European Conf. in}]}
%    \map
%      {\step[fieldsource=booktitle,
%       match=\regexp{In\s+[Pp]roceedings\s+of\s+the.+[Ss]ymposium\s+on}, replace={Symp. on}]}
%    \map
%      {\step[fieldsource=booktitle,
%       match=\regexp{[Pp]roceedings\s+of\s+the.+[Ii]nternational\s+[Cc]onference\s+on}, replace={Intl. Conf. on}]}
%    \map
%      {\step[fieldsource=booktitle,
%       match=\regexp{[Pp]roceedings\s+of\s+the.+[Ii]nternational\s+[Ww]orkshop\s+on}, replace={Intl. Workshop on}]}}}
%

%\theoremstyle{plain}
\newtheorem{theorem}{Theorem}
\newtheorem{lemma}{Lemma}%[section]
\newtheorem{cor}{Corollary}[theorem]
%\theoremstyle{definition}
\newtheorem{definition}{Definition}
\newtheorem{ex}{Example}
%\newproof{proof}{Proof}

% Avoid italics in examples (it's awful to read)
\let\origEx\ex
\let\origendEx\endex
\renewenvironment{ex}{\origEx\upshape}{\origendEx}

% Turn \emph into \textbf inside definitions
\appto{\definition}{\renewcommand{\emph}[1]{\textbf{#1}}}

%color
\definecolor{OliveGreen}{rgb}{0,0.6,0.3}

%References
%% Listings
\def\fref{\Fref} % treat all \frefs as \Frefs
\renewcommand{\lstlistingname}{Snippet}
\newcommand*{\fancyreflstlabelprefix}{lst}
\newcommand*{\Freflstname}{\lstlistingname}
\newcommand*{\freflstname}{\MakeLowercase{\lstlistingname}}
\Frefformat{vario}{\fancyreflstlabelprefix}%
  {\Freflstname\fancyrefdefaultspacing#1#3}
\frefformat{vario}{\fancyreflstlabelprefix}%
  {\freflstname\fancyrefdefaultspacing#1#3}
\Frefformat{plain}{\fancyreflstlabelprefix}%
  {\Freflstname\fancyrefdefaultspacing#1}
\frefformat{plain}{\fancyreflstlabelprefix}%
  {\freflstname\fancyrefdefaultspacing#1}

\renewcommand{\tablename}{Table}
\renewcommand{\figurename}{Figure}

% ln delimiter
\newcommand*{\fancyreflnlabelprefix}{ln}
\newcommand*{\Freflnname}{Line}
\newcommand*{\freflnname}{\MakeLowercase{\Freflnname}}
\Frefformat{vario}{\fancyreflnlabelprefix}%
  {\Freflnname\fancyrefdefaultspacing#1#3}
\frefformat{vario}{\fancyreflnlabelprefix}%
  {\freflnname\fancyrefdefaultspacing#1#3}
\Frefformat{plain}{\fancyreflnlabelprefix}%
  {\Freflnname\fancyrefdefaultspacing#1}
\frefformat{plain}{\fancyreflnlabelprefix}%
  {\freflnname\fancyrefdefaultspacing#1}

%def delimiter
\newcommand*{\fancyrefdeflabelprefix}{def}
\newcommand*{\Frefdefname}{Definition}
\newcommand*{\frefdefname}{\MakeLowercase{\Frefdefname}}
\Frefformat{vario}{\fancyrefdeflabelprefix}%
  {\Frefdefname\fancyrefdefaultspacing#1#3}
\frefformat{vario}{\fancyrefdeflabelprefix}%
  {\frefdefname\fancyrefdefaultspacing#1#3}
\Frefformat{plain}{\fancyrefdeflabelprefix}%
  {\Frefdefname\fancyrefdefaultspacing#1}
\frefformat{plain}{\fancyrefdeflabelprefix}%
  {\frefdefname\fancyrefdefaultspacing#1}

%theorem delimiter
\newcommand*{\fancyreftheolabelprefix}{theo}
\newcommand*{\Freftheoname}{Theorem}
\newcommand*{\freftheoname}{\MakeLowercase{\Freftheoname}}
\Frefformat{vario}{\fancyreftheolabelprefix}%
  {\Freftheoname\fancyrefdefaultspacing#1#3}
\frefformat{vario}{\fancyreftheolabelprefix}%
  {\freftheoname\fancyrefdefaultspacing#1#3}
\Frefformat{plain}{\fancyreftheolabelprefix}%
  {\Freftheoname\fancyrefdefaultspacing#1}
\frefformat{plain}{\fancyreftheolabelprefix}%
  {\freftheoname\fancyrefdefaultspacing#1}

%Proposition delimiter
\newcommand*{\fancyrefproplabelprefix}{prop}
\newcommand*{\Frefpropname}{Proposition}
\newcommand*{\frefpropname}{\MakeLowercase{\Frefpropname}}
\Frefformat{vario}{\fancyrefproplabelprefix}%
  {\Frefpropname\fancyrefdefaultspacing#1#3}
\frefformat{vario}{\fancyrefproplabelprefix}%
  {\frefpropname\fancyrefdefaultspacing#1#3}
\Frefformat{plain}{\fancyrefproplabelprefix}%
  {\Frefpropname\fancyrefdefaultspacing#1}
\frefformat{plain}{\fancyrefproplabelprefix}%
  {\frefpropname\fancyrefdefaultspacing#1}


\renewcommand{\ttdefault}{cmtt}


%%Java definition
\lstdefinelanguage{servalcj}{
	keywords={%Java keywords
	               float, public, interface, class, static, void, extends, implements, final, 
	               boolean, return, new, abstract, super, for, int, package, private, import, 
	               protected, this, throw, try, finally, if, else, while, instance of, &&, ||,
	           %ServalCJ
	               layer, activate, deactivate, when, contextgroup
	               }	
	sensitive=true,
	morecomment=[l]{//},
	morecomment=[s]{/*}{*/},
	morestring=[b]",
}



%JavaScript definition
\lstdefinelanguage{JavaScript}{
keywords={typeof, new, true, false, catch, function, return, null, catch, switch, var, if, in, for, while, do, else, case, break, throw, this, instanceof},
keywordstyle=\color{purple}\bfseries\ttfamily,
ndkeywords={},
ndkeywordstyle=\color{blue}\bfseries,
identifierstyle=\color{black},
sensitive=false,
comment=[l]{//},
morecomment=[s]{/*}{*/},
commentstyle=\color{OliveGreen}\ttfamily,
stringstyle=\color{OliveGreen}\ttfamily,
morestring=[b]',
morestring=[b]"
}

\lstset{%
  basicstyle=\scriptsize,
  aboveskip=0\baselineskip,
  belowskip=0\baselineskip,
  commentstyle=\color{gray}\footnotesize\ttfamily\itshape,
  prebreak= ,
  numberblanklines=false,
  breaklines,
  numberstyle=\tiny\color{gray},
  numbersep=0pt,
  escapechar=`}

\lstdefinestyle{floating}{%
  frame=none,
  float=htb,
  captionpos=b,
  aboveskip=0pt,
  belowskip=0pt
}

% context traits listings
\lstdefinestyle{ctxtraits}
 {language=JavaScript,
  frame=lines,
  showstringspaces=false,
  tabsize=3,
  style=floating,
  morekeywords={Trait, cop, Context, activate, deactivate, adapt, addObjectPolicy, manager, for:, activate:, deactivate:, and, or, unique, allOf, atLeastOne, between, atMostOne, in}
}

%context traits environment
\lstnewenvironment{ctxtraits}[1][]
 {\lstset{style=ctxtraits,#1}}{}

 % Context Traits in line source-code
\newcommand{\sctxtraits}[1]{\textrm{\texttt{#1}}}
\def\scode{\lstinline[style=ctxtraits]}

% ContextJ/ServalCj listings
\lstdefinestyle{contextj}
 {language=Java,
  frame=lines,
  showstringspaces=false,
  keywordstyle=\ttfamily\bfseries,
  tabsize=3,
  style=floating,
  morekeywords={with, when, global, activate}
}

%ContextJ/ServalCj environment
\lstnewenvironment{contextj}[1][]
 {\lstset{style=contextj,#1}}{}

% ContextJ/ServalCj in line source-code
\newcommand{\scj}[1]{\textrm{\texttt{#1}}}


% Subjective-C listings
\lstdefinestyle{subc}
 {language=C,
  frame=lines,
  showstringspaces=false,
  keywordstyle=\ttfamily\bfseries,
  tabsize=3,
  style=floating,
  morekeywords={@activate, @deactivate, @context, @contexts, @active}
}

%Subjective-C environment
\lstnewenvironment{subc}[1][]
 {\lstset{style=subc,#1}}{}

% Subjective-C in line source-code
\newcommand{\ssubc}[1]{\textrm{\texttt{#1}}}


%----[ Figures ]---
%\addtolength{\intextsep}{-1mm}

%%captions
%\addtolength{\abovecaptionskip}{-0mm}
%\addtolength{\belowcaptionskip}{-0mm}
%\captionsetup[figure]{aboveskip=-0ex,belowskip-0ex}
%\captionsetup[table]{aboveskip=-0ex,belowskip-0ex}
%\captionsetup[lstlisting]{aboveskip=-1ex,belowskip=-1ex}

%DPN
\usepackage[version=0.96]{pgf}
\usepackage{tikz}
\usetikzlibrary{arrows, shapes, backgrounds}
\usetikzlibrary{decorations.pathreplacing}
\usetikzlibrary{shapes.misc}
\usetikzlibrary{petri}

%% Petri nets
\tikzstyle{place}=[circle,thick,draw=black!75,minimum size=5mm]
\tikzstyle{iplace}=[circle,dashed,thick,draw=black!75,minimum size=5mm]
\tikzstyle{itransition}=[rectangle,draw,thick,fill=black,minimum size=1mm]
\tikzstyle{etransition}=[rectangle,draw,thick,minimum size=1mm]
\tikzstyle{ctransition}=[rectangle,draw,color=black!45,thick,fill=black!45,minimum size=1mm]

\tikzstyle{dpn}=
 [node distance=1.3cm, >=stealth', bend angle=45, auto,
  font=\fontsize{8}{8}\selectfont]

%----[ Commands ]---
%Latins
\newcommand{\eg}{\emph{e.g.,}\xspace}
\newcommand{\etal}{\emph{et al.}\xspace}
\newcommand{\ie}{\emph{i.e.,}\xspace}
\newcommand{\cf}{\emph{cf.}\xspace}

\newcommand{\ctx}[1]{\texttt{\textsc{#1}}}

%petri
\newcommand{\enabled}[2]{$#1[#2\rangle$}


%comments
% xcolor
\definecolor{author}{rgb}{.5, .5, .5}
\definecolor{comment}{rgb}{.1, .0, .9}
\definecolor{note}{rgb}{.9, .4, .0}
\definecolor{idea}{rgb}{.1, .7, .0}
\definecolor{missing}{rgb}{.9, .1, .0}


\newcommand{\authorcomment}[3][comment]
  {\ifdraft{\noindent
      \fbox{\footnotesize\textcolor{author}{\textsc{#2}}}
      \textcolor{#1}{\textsl{#3}}}{}}

\input{acronyms}

%Space squeezing
\let\orig@figure\figure
\renewcommand*{\figure}[1][]{\orig@figure[#1]\vspace{-0ex}} % before
\let\orig@endfigure\endfigure
\renewcommand*{\endfigure}{\vspace{-0ex}\orig@endfigure} % after

% Squeeze captions
\usepackage{caption}
\captionsetup[figure]{aboveskip=0.0em,belowskip=-0em}
\captionsetup[table]{aboveskip=0em,belowskip=-0em}

%=========================================PLEGER
%Helper commands
\definecolor{red}{rgb}{1,0,0}
\newcommand\change[1]{\textcolor{red}{#1}}
\newcommand{\kw}[1]{{\bf#1}}
\newcommand{\parhead}[1]{\noindent{\bf {\em #1.}}}

\newcommand{\mynote}[2]{
\fbox{\bfseries\sffamily\scriptsize#1}
  {\small\textsf{\emph{#2}}}
\fbox{\bfseries\sffamily\scriptsize }
}

\newcommand{\co}[1]{\texttt{\textsc{#1}}}
\newcommand{\super}[1]{\ensuremath{^{\textrm{#1}}}}
\newcommand{\sub}[1]{\ensuremath{_{\textrm{#1}}}}

%Proposal/languages
\newcommand{\java}[0]{\mbox{Java}\xspace}
\newcommand{\javascript}[0]{\mbox{JavaScript}\xspace}
\newcommand{\csi}[0]{\mbox{CSI}\xspace}
\newcommand{\raijs}[0]{\mbox{RI\--JS}\xspace}
\newcommand{\ema}[0]{\mbox{\ac{EMA}}\xspace}
\newcommand{\emajs}[0]{\mbox{EMAjs}\xspace}
\newcommand{\servalcj}[0]{\mbox{ServalCJ}\xspace}
%\newcommand{\ai}[0]{\mbox{AI}}

%Web
\newcommand{\ecmascript}[0]{\mbox{ECMAScript}\xspace}
\newcommand{\nodejs}[0]{\mbox{Nodejs}\xspace}

%Some comments
\newcommand\todo[1]{\mynote{TODO}{#1}}
\newcommand\fix[1]{\mynote{FIX}{#1}}
\newcommand\reform[1]{\mynote{reform}{#1}} %reformulate
\newcommand{\HERE}[0]{\bigskip \bigskip \mbox{\bf**************HERE*************}\bigskip}


%Personal comments
\newcommand\pl[1]{\mynote{PL}{#1 }}
\newcommand\nc[1]{\mynote{NC}{#1 }}


%\makeatother
